\documentclass[11pt]{article}
\usepackage[utf8]{inputenc}
\usepackage[LGR,T1]{fontenc}
\usepackage[french]{babel}
\usepackage{amsmath}
\usepackage{listings}
\usepackage{xcolor}
\usepackage{geometry}
\usepackage{amssymb}
\usepackage{graphicx}
\usepackage{tikz}

%pour pouvoir mettre du grec
\newcommand{\textgreek}[1]{\begingroup\fontencoding{LGR}\selectfont#1\endgroup}

%marges
\geometry{hmargin=3cm,vmargin=3cm}

\title{Projet MOPSI}
\author{Louis Hémadou et Louis Lesueur}
\date{}

%chemin pour les images:
%\graphicspath{{Images/}}

\definecolor{darkWhite}{rgb}{0.94,0.94,0.94}

%pour insérer du code avec l'environnemet lstlisting
\lstset{
  aboveskip=3mm,
  belowskip=-2mm,
  backgroundcolor=\color{darkWhite},
  basicstyle=\footnotesize,
  breakatwhitespace=false,
  breaklines=true,
  captionpos=b,
  commentstyle=\color{red},
  deletekeywords={...},
  escapeinside={\%*}{*)},
  extendedchars=true,
  framexleftmargin=16pt,
  framextopmargin=3pt,
  framexbottommargin=6pt,
  frame=tb,
  keepspaces=true,
  keywordstyle=\color{blue},
  language=python,
  literate=
  {²}{{\textsuperscript{2}}}1
  {⁴}{{\textsuperscript{4}}}1
  {⁶}{{\textsuperscript{6}}}1
  {⁸}{{\textsuperscript{8}}}1
  {€}{{\euro{}}}1
  {é}{{\'e}}1
  {è}{{\`{e}}}1
  {ê}{{\^{e}}}1
  {ë}{{\¨{e}}}1
  {É}{{\'{E}}}1
  {Ê}{{\^{E}}}1
  {û}{{\^{u}}}1
  {ù}{{\`{u}}}1
  {â}{{\^{a}}}1
  {à}{{\`{a}}}1
  {á}{{\'{a}}}1
  {ã}{{\~{a}}}1
  {Á}{{\'{A}}}1
  {Â}{{\^{A}}}1
  {Ã}{{\~{A}}}1
  {ç}{{\c{c}}}1
  {Ç}{{\c{C}}}1
  {õ}{{\~{o}}}1
  {ó}{{\'{o}}}1
  {ô}{{\^{o}}}1
  {Õ}{{\~{O}}}1
  {Ó}{{\'{O}}}1
  {Ô}{{\^{O}}}1
  {î}{{\^{i}}}1
  {Î}{{\^{I}}}1
  {í}{{\'{i}}}1
  {Í}{{\~{Í}}}1,
  morekeywords={*,...},
  numbers=left,
  numbersep=10pt,
  numberstyle=\tiny\color{black},
  rulecolor=\color{black},
  showspaces=false,
  showstringspaces=false,
  showtabs=false,
  stepnumber=1,
  stringstyle=\color{gray},
  tabsize=4,
  title=\lstname,
}


\begin{document}
    \maketitle
    \section{Exemple en 1D}
    \paragraph{}
    On considère $\Omega = [0,1]$ et $\rho_1, \rho_2 \in \mathcal{P}(\Omega)$ \\
    Les densités $\rho_1$ et $\rho_2$ sont discrétisées (cf. pixelisation)
    
    \vspace{0.5cm}
    
    \begin{tikzpicture}
    \draw (0,0) -- (4,0);
    \draw (0,0.1cm) -- (0,-0.1cm) node[below] {$x_0=0\strut$};
    \draw (0.5,0.1cm)  (0.5,-0.1cm) node[above] {$\Delta x = 1/M \strut$};
    \draw (1,0.1cm) -- (1,-0.1cm) node[below] {$x_1\strut$};
    \draw (2,0.1cm) -- (2,-0.1cm) node[below] {$...\strut$};
    \draw (3,0.1cm) -- (3,-0.1cm) node[below] {$...\strut$};
    \draw (4,0.1cm) -- (4,-0.1cm) node[below] {$x_M=1\strut$};
    \end{tikzpicture}
    \vspace{0.5cm}
    
    On pose donc: $\rho_i(x) = \rho_i^j$ pour chaque $x \in [x_{j-1}, x_j[$ ( densités de proba constantes par morceaux, telles que $\Delta x \sum_{j=1}^{M} \rho_i^j  =1$ )
    
    \paragraph{Remarque} Pour une image, l'espace sera en 2D ( $N_y$ pixels de haut, $N_x$ pixels de large ). Dans le cas d'une image en noir et blanc 1=Noir, 0=blanc et gris entre 0 et 1.
    
    \subsection{Formalisation du problème}
    
    \paragraph{}
    On cherche une sorte d'interpolation entre $\rho_1$ et $rho_2$, une sorte de $\lambda \in [0,1]$ qui permettrait de trouver une densité du type $(1-\lambda) \rho_1 + \lambda \rho_2$. Mais c'est plus compliqué, donc on va remplacer le "$(1-\lambda)$" par une équation.
    
    Pour ça, on se fixe un $\epsilon > 0$ et on se donne $W_\epsilon$ une distance sur $\mathcal{P}(\Omega)$ et on va chercher un barycentre pour cette distance. C'est-à-dire un  $\bar{\rho_\lambda}$ tel que pour $\lambda \in [0,1]$ \footnote{cf formule (8) du papier}
    \begin{equation}
    \bar{\rho_\lambda} \in \underset{\rho \in \mathcal{P}}{argmin} \hspace{0.2cm} \lambda W_\epsilon(\rho, \rho_1)^2 + (1-\lambda) W_\epsilon(\rho, \rho_2)^2
    \end{equation}
    
    \paragraph{Remarque}
     On peut vérifier que pour $\lambda=1$, $\rho_1$ est bien solution et que pour $\lambda=0$, $\rho_2$ est aussi solution.
    
    \paragraph{Exercice}
    Si on choisit une distance induite par une norme pour $W_\epsilon$ dans un espace vectoriel normé ( ce qui n'est pas le cas de $\mathcal{P}(\Omega)$ ! ), montrer que $\bar{\rho_\lambda}$ est unique et que $\bar{\rho_\lambda} = \lambda \rho_1 + (1-\lambda) \rho_2$, ce qui justifie le nom de barycentre !
    
    \newpage
    \subsection{Distance de Wasserstein régularisée}
    
    Pour le $W_\epsilon$ on va considérer la distance de Wasserstein régularisée\footnote{cf section 3.1 du papier}:
    
    \begin{equation}
    W_\epsilon(\rho_1, \rho_2) = \underset{\pi \in \mathcal{P}(\Omega \times \Omega )}{min} \hspace{0.2cm} \int_{\Omega \times \Omega} (x-y)^2 \pi(x,y) dx dy + \epsilon KL(\pi | \xi )
    \end{equation}
    
    Avec:
    \begin{itemize}
    \item $KL(\pi | \xi ) = \int_{\Omega \times \Omega} \pi(x,y) (\log (\frac{\pi(x,y)}{\xi(x,y)}) -1) dx dy$
    \item $\xi \in \mathcal{P}(\Omega \times \Omega )$ mesure de proba de référence ( on peut choisir $\xi=1$ ou $\xi(x,y)= \rho_1(x) \rho_2(y)$ par exemple )
    \end{itemize}
     
    
    \paragraph{Propriétés de $\pi \in \mathcal{P}(\Omega \times \Omega )$}
    \begin{itemize}
    \item $\pi(x,y) \geq 0$
    \item $\int_{\Omega \times \Omega} \pi(x,y) dx dy=1$
    \item $\int_{\Omega \times \Omega} \pi(x,y) dy=\rho_1(x)$ est la première marginale
    \item $\int_{\Omega \times \Omega} \pi(x,y) dx =\rho_2(y)$ est la seconde marginale
    \end{itemize}
    
    \subsection{Retour sur le cas discret sur $[0,1]$}
    \paragraph{}
    On pose, pour $1 \leq j \leq M$: $\tilde{x_j}=(j+\frac{1}{2}) \Delta x$
    
    \paragraph{}
    À partir d'ici on peut abandonner les "$\Delta x$" grâce aux simplexes, comme dans le papier.
    
    \subsection{Discrétisations des outils définis ci-dessus}
    \paragraph{}
    On définit $(\pi_{ij})_{1 \leq i,j \leq M}$ tel que:
    \begin{itemize}
    \item $\pi_{j,j'} \geq 0$
    \item $\sum_{j,j'} \pi_{jj'} =1$
    \item $\sum_{j'} \pi_{jj'} = \rho_1^j$
    \item $\sum_j \pi_{jj'} = \rho_2^{j'}$
    \end{itemize}
    
    \paragraph{}
    Norme de Wasserstein discrétisée:
    \begin{equation}
    W_\epsilon(\rho_1, \rho_2) = \sum_{1 \leq j,j' \leq M} c_{jj'} \pi{jj'} + \epsilon \sum_{1 \leq j,j' \leq M} \pi_{jj'} ( \log(\frac{\pi_{jj'}}{\xi_{jj'}})-1)
    \end{equation}
    avec $c_{jj'}=(\tilde{x_j} - \tilde{x_{j'}})^2$ \textcolor{red}{Faut-il mettre des $\Delta x$ ? En reparler à Virginie quand on rencontrera le problème.}
    
    Cette norme se calcule avec la section 2 du papier
    
    \newpage
    \section{À faire pour la prochaine fois}
    \begin{itemize}
    \item Lire le papier jusqu'à la page 11
    \item Implémenter un algo pour générer des densités de proba discrètes
    \item Calculer des $W_\epsilon$ et regarder ce qu'il se passe pour différentes valeurs de $\epsilon$ ( reproduire ce qu'il se passe sur la Figure 1 du papier où $\rho_1$ est en bleu et $\rho_2$ est en rouge ) 
    \item prochain rdv le $07/11$ à $17^h 30$
    \end{itemize}
    
    
    
    
    
  

\end{document}
